\documentclass[a4paper,12pt]{article}
\usepackage[utf8]{inputenc}
\usepackage[ngerman]{babel}
\usepackage{setspace}
\usepackage{parskip}
\usepackage{enumitem}
\usepackage[left=1in, right=1in, top=1in, bottom=1in]{geometry}
\setstretch{1.2}
\setlength{\parskip}{1em}

\title{Stellungnahme gegen den Verlust des Prüfungsanspruchs der Bachelor- und Masterstudiengänge der Fakultät Mathematik und Informatik}
\author{}
\date{}

\begin{document}

\maketitle

Der Verlust des Prüfungsanspruchs für meinen Bachelorstudiengang im Bereich Informatik stellt für mich eine einschneidende und belastende Situation dar.
Auch wenn ich aktuell mein Studium nicht weiterführe, sehe ich diesen Schritt mehr als Unterbrechung anstatt eines endgültigen Abbruches an.
Der Bachelorabschluss bleibt ein Ziel, das ich, nach meiner geplanten Ausbildung zum Fachinformatiker für Systemintegration, abschließen möchte.
Deshalb möchte ich mich entschieden gegen den Verlust des Prüfungsanspruchs aussprechen und eine Möglichkeit schaffen, diese Option offenzuhalten.

%Das Informatikstudium hat für mich trotz aller Herausforderungen eine besondere Bedeutung.
Trotz aller Herausforderungen ist das Informatikstudium weiterhin teil meines zukünftigen Werdegangs.
Mein Interesse für Informatik bleibt ungebrochen, auch wenn ich festgestellt habe, dass das Studium aktuell nicht den richtigen Weg für mich darstellt.
Diese Erkenntnis hat mich zu einer beruflichen Neuorientierung bewogen.
Dennoch halte ich es für wichtig, langfristig eine akademische Perspektive offenzuhalten.
% Der Verlust des Prüfungsanspruchs würde diese Tür dauerhaft schließen und mir eine bedeutende Möglichkeit zur Weiterentwicklung nehmen.
\\

\begin{itemize}[label={-}, labelwidth=2em, left=1em]
\item \textbf{Persönliche und berufliche Neuorientierung:}\\
Ab dem 01.02.2025 werde ich bei der Encevo Deutschland GmbH als IT-Aushilfe tätig sein und dort ebenfalls ab dem 01.09.2025 eine Ausbildung zum Fachinformatiker für Systemintegration beginnen.
Diese Umorientierung bedeutet kein Abschied von der Informatik, sondern vielmehr eine Fokusierung auf den Praxisteil einer Ausbildung.
Hierdurch gewinne ich nicht nur wertvolle Erfahrungen, sondern baue auch weitere Fachkenntnisse auf, die ich später in einer Fortsetzung meines Studiums vertiefen als auch anwenden könnte.

\break
\item \textbf{Zukunftsoptionen und berufliche Perspektiven:}\\
\textbf{Rework}\\
Die Digitalisierung und die zunehmende Bedeutung von IT in nahezu allen Lebensbereichen machen eine akademische Qualifikation in Informatik auch langfristig attraktiv.
Sollte ich nach meiner Ausbildung und ersten Berufserfahrungen feststellen, dass ein Bachelorabschluss in Informatik für meine Karriereplanung von Vorteil ist, möchte ich diese Option unbedingt wahrnehmen können.
Der Verlust des Prüfungsanspruchs würde mir diese Perspektive nehmen und meinen Handlungsspielraum erheblich einschränken.

\item \textbf{Bereits erbrachte Leistungen und bisheriges Engagement:}\\
Im Verlauf meines Studiums habe ich bereits Zeit und Energie in die Bearbeitung der Studieninhalte investiert.
Auch wenn ich mein Studium aktuell nicht weiterführe, stellen die bisherigen Leistungen einen wichtigen Grundstein dar, auf den ich später aufbauen möchte.
Der Verlust des Prüfungsanspruchs würde jedoch bedeuten, dass ich diese erbrachten Leistungen nicht mehr nutzen kann, was aus meiner Sicht unverhältnismäßig wäre.

\item \textbf{Flexibilität und individuelle Lebenswege:}\\
Bildungswege verlaufen selten geradlinig, und es sollte Raum für individuelle Entwicklungen geben.
Mein aktueller Weg ist nicht das Ende meiner akademischen Laufbahn, sondern ein Umweg, der mir neue Perspektiven eröffnet.
Es wäre daher im Sinne von Flexibilität und individueller Lebensgestaltung, den Prüfungsanspruch aufrechtzuerhalten, um mir einen späteren Wiedereinstieg ins Studium zu ermöglichen.
\\
\end{itemize}

\section*{\normalsize Aktuelle und zukünftig geplante Leistungen}
Neben meiner Tätigkeit als IT-Aushilfe ab Februar 2025 werde ich mich im Rahmen der Ausbildung aktiv und mit großem Engagement weiterbilden.
Mein Ziel ist es, mein bereits vorhandenes Wissen mit neuen praktischen und theoretischen Kenntnissen zu verknüpfen und damit eine solide Grundlage für einen möglichen Wiedereinstieg ins Studium zu schaffen.
% Auch plane ich, mich parallel zur Ausbildung autodidaktisch mit Informatikthemen zu beschäftigen, um meinen Wissensstand kontinuierlich zu erweitern.

Zusammenfassend möchte ich betonen, dass der Verlust des Prüfungsanspruchs meine beruflichen und akademischen Perspektiven erheblich einschränken würde.
Angesichts meiner Neuorientierung, der bereits erbrachten Studienleistungen und der Relevanz eines möglichen Bachelorabschlusses für meine berufliche Zukunft plädiere ich dafür, den Prüfungsanspruch beizubehalten.
Eine Lösung könnte darin bestehen, eine Wiedereinstiegsmöglichkeit nach Abschluss der Ausbildung und unter bestimmten Bedingungen zu gewähren.
%Dies würde nicht nur mir, sondern auch anderen Studierenden in ähnlichen Situationen gerecht werden und die notwendige Flexibilität in unserem Bildungssystem unterstreichen.

\end{document}