\section{Komponenten eines Arbeitsplatzcomputers unterscheiden}
\subsection{Zentraleinheit, Mainboard und Betriebssystem unterscheiden}
    \begin{subindent}
        Die Zentraleinheit bezeichnet das Mainboard mit den darauf angebrachten Komponenten:
        \begin{itemize}[leftmargin=2.5cm, topsep=0.2em, itemsep=0.1em, parsep=0.3em]
            \item Prozessor\@: wichtigste Komponente, Verarbeitung der Daten findet hier statt (s. 2.4.3)
            \item Arbeistspeicher\@: flüchtiger Zwischenspeicher zum schnellen Zugriff durch die CPU (s. 2.4.4)
            \item ROM\@: Read-Only-Memory, enthält Grundbetriebssytem\@: \\
                  BIOS: Basic-Input-Output-System, Firmware für Start des eigentlichen Betriebssystems und Basissteuerung der Hardware \\
                  UEFI: Unified Extensible Firmware Interface, eigenes Betriebssystem, wodurch Updates geladen werden können \\
                  UEFI löst das `veraltete' BIOS ab
        \end{itemize}
        Nach aufruf des ROM wird das Betriebssystem gestartet. Dies führt die GUI (Bedienoberfläche) aus und wird von Takten, die durch den Taktgeber innerhalb der CPU (Clock) gegeben werden, bearbeitet.
        Das Betriebssystem führt verschiedene Anwenderprogramme aus. Da Kompatibilität eines Programms nicht für jedes Betriebssystem gegeben ist, können Virtual Machine Monitors (VMMs) verwendet werden.
        Auf diesen kann ein Gastbetriebssystem installiert werden, welches virtuell die benötigten Hardwareressourcen zur Verfügung bekommt. Dies nennt man auch Betriebssystem-Virtualisierung und erfordert die Installation eines Virtualisierungs-Hypervisor.
    \end{subindent}

    \begin{figure}[H]
        \centering
        \includegraphics[width=0.5\textwidth]{./images/2.4.1_computerkonfig.png}
        \caption{Beispiel Computerkonfiguration}\label{fig:Computerkonfiguration}
    \end{figure}

\subsection{Hauptplatine, Mainboard und die Komponenten unterscheiden}
    \begin{figure}[H]
        \centering
        \includegraphics[width=0.5\textwidth]{./images/2.4.2_mainboard.png}
        \caption{Mainboard}\label{fig:Mainboard}
    \end{figure}
    
    \begin{subindent}
        \begin{itemize}[leftmargin=2.5cm, topsep=0.2em, itemsep=0.1em, parsep=0.3em]
            \item \textbf{Mainboard:} \\
                  auch Motherboard oder Systemplatine, Hauptplatine, auf der alle Komponenten angebracht sind
            \item \textbf{BIOS:} \\
                  zuständig für Startvorgang, enthält in \textbf{EPROM} ein Basisbetriebssystem
            \item \textbf{Chipsatz:} \\
                  zuständig für Kommunikation der Komponenten untereinander
            \item \textbf{Sockel:} \\
                  physikalische Verbindung von Mainboard und Prozessor
            \item \textbf{Peripherie-Anschlüsse, (PCIe-)Steckplätze:} \\
                  I/O-Peripherie für externe Hardware (z.B. Maus/Tastatur) \\
                  Steckplätze auf Mainboard für interne Hardware (z.B. RAM, GPU oder SATA/M.2)
            \item \textbf{Netzteil:} \\
                  Stromversorgung aller Komponenten
        \end{itemize}
    \end{subindent}
    
    \begin{figure}[H]
        \centering
        \includegraphics[width=0.7\textwidth]{./images/2.4.2_formfaktoren.png}
        \caption{Formfaktoren}\label{fig:Formfaktoren}
    \end{figure}

\subsection{Prozessoren genauer beschreiben}
    \begin{subindent}
        Der Mikroprozessor ist die wichtigste Komponente. Sie ist nur so schnell wie die schwächste, mit ihr verbundene Komponente auf dem Motherboard.
        Die größten Prozessorhersteller sind AMD und Intel, welche sich im ständigen Wettbewerb um die Leistungstärkste CPU befinden. \\
        Begriffserklärung\@: \\
        Der Mikroprozessor (CPU = Central Processing Unit) holt Befehle aus dem Zwischenspeicher und steuert mit der Ausführung dieser die Verarbeitung, Steuerung und Kontrolle von Daten und Systemen.
    \end{subindent}

    \begin{tcolorbox}[width=15cm, center, title=CPU Begriffe, coltitle=white, colframe=orange, colback=white!60!orange]
        \begin{itemize}[itemsep=0.1em, parsep=0.3em]
            \item Bestandteile\@: \\
                  ALU (Arithmetic Logic Unit)\@: Rechenwerk \\
                  CU (Control Unit)\@: Steuereinheit \\
                  MMU (Memory Management Unit)\@: Speichermanager \\
                  CPU-Cache\@: Zwischenspeicher
            \item Prozessorarchitekturen\@: erhalten Codenamen von den Herstellern
            \item Strukturgröße\@: bei Intel 7nm, bei AMD bereits 5nm; je kleiner die Angabe desto leistungstärker die CPU
            \item Cache\@: Schneller Zwischenspeicher der CPU entlastet (s. 2.4.4)
            \item Chipsatz\@: Halbleiterbaustein der die Komponenten untereinander verbindet, Anzahl der anzubindenden Komponenten daher sehr wichtig
            \item Erhöhung der Taktfrequenz\@: die maximale Taktfrequenz kann durch Übertaktung erhöht werden, ist mit Erhöhung des Energieverbrauchs, Wärmeentwicklung und Kühlerlautstärke verbunden
            \item Parallelisierung\@: \\
                  Multi-Threading\@: mehrere Programmabläufe (Threads) gleichzeitig \\
                  Pipelining\@: parallele Befehlsausführung \\
                  Coprozessor\@: Entlastung bei bestimmten Aufgaben und Funktionen durch spezielle CPU \\
                  Multi-Prozessor\@: mehrere CPUs, selten bei PCs (eher Industrieanwendung) \\
                  Multi-Core-Prozessor\@: mehrere Cores (Rechenkerne) innerhalb der CPU
            \item Turbo Boost/Precision Boost\@: automatische Übertaktung bei Multi-Core-CPUs ohne negativen Effekte der aktiven Übertaktung (unterhalb der TDP-Grenze)
            \item Multimedia-Erweiterung: Hersteller stellen Befehlsätze (MMX, XMM, SEE) zur Verfügung um höhere Anforderungen (z.B. Spielsoftware) zu entsprechen
            \item TDP\@: Thermal Design Power, maximale Wärmeleistung der CPU
            \item GPGPU\@: General Purpose Computation on Graphics Processing Unit, Programmierschnittstelle um leistungsintensive Berechnungen auf Grafikkarte auszuführen
        \end{itemize}
    \end{tcolorbox}
    
\subsection{Arbeistspeicher (RAM-Speicher) erläutern}
    \begin{subindent}
        RAM-Speicher (Random Access Memory) ist ein flüchtiger Arbeitsspeicher, über den die CPU auf Daten zugreift, wenn mehrere Programme parallel benutzt werden. Da RAM flüchtig ist werden die Daten beim Herunterfahren des PCs gelöscht. \\
        RAM-Formate:
        \begin{itemize}[leftmargin=2.5cm,, itemsep=0.1em, parsep=0.3em]
            \item DIMM\@: Dual In Line Memory Module, wird in Desktops und Servern verwendet
            \item SO-DIMM\@: Small Outline DIMM, wird in Laptops verwendet
        \end{itemize}
        Neben dem Arbeitsspeicher gibt es noch den (im Vergleich zur RAM) schnellen Cache-Speicher. \\
        Cache-Levels:
        \begin{itemize}[leftmargin=2.5cm, itemsep=0.1em, parsep=0.3em]
            \item L1-Cache: Geschwindigkeit ähnlich zu Prozessor, für häufig verwendete Befehle und Daten
            \item L2-Cache: größer und langsamer als L1, aber schneller als RAM
            \item L3-Cache: Datenabgleich Caches und Cores
        \end{itemize}
        Die Geschwindigkeiten werden in Megatransfers per Second (MT/s) angegeben.
    \end{subindent}

    \begin{figure}[H]
        \centering
        \includegraphics[width=0.7\textwidth]{./images/2.4.4_ramspeeds.png}
        \caption{RAM-Geschwindigkeiten}\label{fig:RAM-Geschwindigkeiten}
    \end{figure}

    \begin{tcolorbox}[width=15cm, center, title=RAM Begriffe, coltitle=white, colframe=orange, colback=white!60!orange]
        \begin{itemize}[itemsep=0.1em, parsep=0.3em]
            \item RAM\@: Random Access Memory
            \item JEDEC\@: \\ Joint Electronic Device Engineering Council \\ Organisation legt Spezifikationen für elektrische und zeitliche Parameter der Speichercontroller und -chips fest
            \item Formfaktoren: \\ UDIMM (Unbuffered DIMM): häufigstes Format in Desktops \\ SO-DIMM\@: kleiner und physikalisch kürzer als UDIMMS
            \item DRAM\@: Dynamic Random Access Memory \\ Jedes Datenbit wird auf seperatem Kondensatoren gespeichert
            \item SDRAM\@: Synchronous Dynamic Random Access Memory \\ getaktetes DRAM, Daten werden synchron zum Speicher-Bus übertragen
            \item DDR-RAM\@: Double Data Rate RAM \\ überträgt Daten doppelt so schnell wie SDRAM\@; neueste Generation ist DDR5 (Gens untereinander nicht kompatibel)
            \item DDR-SDRAM\@: Weiterentwicklung der SDRAM-Technologie
            \item SSD-RAM\@: Solid State RAM \\ flash-based Speicher, SSD-Speicher wird als zusätzliche RAM benutzt, die Daten darauf bleiben beim Herunterfahren erhalten
            \item QLC\@: Quad-Level-Cells \\ neueste Tech der Flash-Speicherarchitektur, speichert vier Datenbits in jeder Datenzelle
            \item FSB\@: Frontsidebus \\ Hauptpfad für Daten im Computer, verbindet CPU, DRAM, GPU und Chipsatz
            \item Latenz\@: Zeit, die Speicher benötigt, um auf Befehl zu reagieren
            \item ECC\@: Error Correcting Code \\ teures `fully buffered, registered ECC RAM', das hilft Speicherfehler zu minimieren oder selbst zu korrigieren 
        \end{itemize}
    \end{tcolorbox}

\subsection{Schnittstellen und Anschlüsse am Mainboard erläutern}
    \begin{figure}[H]
        \centering
        \includegraphics[width=0.7\textwidth]{./images/2.4.5_anschluesse.png}
        \caption{Mainboard Anschlüsse}\label{fig:Mainboard Anschlüsse}
    \end{figure}

    \begin{tcolorbox}[width=15cm, center, title=Anschlüsse am Mainboard, coltitle=white, colframe=orange, colback=white!60!orange]
        \begin{itemize}[itemsep=0.1em, parsep=0.3em]
            \item Sockel\@: Ort in dem der Prozessor platziert wird
            \item RAM-Steckplätze
            \item PCI(e)-Steckplätze\@: Peripheral Component Interconnect Express (PCIe) \\ Anschluss von Zusatzkarten, verfügbare Steckplätze sind abhängig vom Motherboard
            \item SATA-Anschlüsse\@: Verbindung von Festplatten und Laufwerken \\ Blaue SATA-Buchsen: 6 GBit/s, Abwärtskompatibel \\ Schwarze/Rote SATA-Buchsen: Herstellerabhängige Bedeutung
            \item USB-Schnittstellen\@: Schnittstellen für USB-Ports
            \item M.2-Port\@: Anschluss für SDD
            \item Lüfteranschluss
            \item Stromanschluss
            \item Monitoranschlüsse
            \item USB-Anschlüsse
            \item P/S2-Port\@: Anschluss für ältere Mäuse und Tastaturen
            \item LAN
            \item Klinkenanschlüsse\@: Tonübertragung
        \end{itemize}
    \end{tcolorbox}

    \begin{figure}[H]
        \centering
        \includegraphics[width=0.8\textwidth]{./images/2.4.5_pciespeeds.png}
        \caption{PCIe Geschwindigkeiten}\label{fig:PCIe Geschwindigkeiten}
    \end{figure}

    \begin{figure}[H]
        \centering
        \includegraphics[width=0.7\textwidth]{./images/2.4.5_usbspeed.png}
        \caption{USB Geschwindigkeiten}\label{fig:USB Geschwindigkeiten}
    \end{figure}

\subsection{Netzteile beschreiben und unterscheiden}
    \begin{subindent}
        Netzteile (PSU - Power Supply Unit) versorgen den PC mit Strom. Sie wandeln Wechselstrom in den benötigten Gleichstrom um. \\
        Vorraussetzungen für passende Netzteile:
    \end{subindent}

    \begin{itemize}[leftmargin=2.5cm, topsep=0.2em, itemsep=0.1em, parsep=0.3em]
        \item passender Formfaktor (ATX, SFX, TFX, ITX)
        \item genug Watt um alle Komponenten zu versorgen (ca. 120 bis 1800)
        \item Kühlungsmöglichkeiten\@: \\ -passiv: geräuschlos, aber niedrige Wattleistung \\ -aktiv: eingebauter Lüfter, mehr Leistung
    \end{itemize}

\subsection{Festplatten unterscheiden und erläutern}
    \begin{subindent}
        Zum Erhalt von Daten nach dem Herunterfahren werden externe Speicher benötigt. Diese Festplatten müssen leistungsmäßig an die anderen Komponenten des Systems angepasst werden.
        Meist werden erwerbbare Festplatten vorformatiert, d.h. Spuren, Sektoren und Dateisystem kommen voreingerichtet. Primäres Dateisystem auf Windows: NTFS
    \end{subindent}

    \begin{tcolorbox}[width=15cm, center, title=Arten von Festplatten, coltitle=white, colframe=orange, colback=white!60!orange]
        \begin{itemize}[itemsep=0.1em, parsep=0.3em]
            \item HDD\@: Hard Disk Drive \\ Dateispeicher über Schreib-Leseköpfe an drehenden, magnetischen Metallscheiben \\ langsamer, billiger als SSD
            \item SDD\@: Solid State Drive \\ unbeweglicher Block als Flash-Speicher \\ bis zu 3x so schnell wie HDDs, geräuschlos
            \item SSHD\@: Hybrid-Laufwerke \\ Daten auf HDD, Speicherung zur Verarbeitung auf SSD \\ bis zu 5x so schnell wie HDDs
            \item USB-Sticks\@: Flash-Speicher \\ klein, leise und mobil
        \end{itemize}
    \end{tcolorbox}

    \begin{tcolorbox}[width=15cm, center, title=Spezifikationen von Festplatten, coltitle=white, colframe=orange, colback=white!60!orange]
        \begin{itemize}[itemsep=0.1em, parsep=0.1em]
            \item Bauform\@: \\ HDD\@: 1,8/2,5 Zoll für Notebooks; sonst 3,5 Zoll \\ SSD\@: intern als Speicherriegel, extern 2,5/3,5 Zoll
            \item Performance\@: abhängig von mittlerer Zugriffszeit, Datentransferrate und Schnittstellenleistung
            \item Umdrehungsgeschwindigkeit HDD\@: bis zu 15000rpm (Standard: 7200rpm)
            \item Datentransferrate\@: siehe nachfolgende Tabelle
            \item Cache\@: eingebauter Zwischenspeicher
            \item Partitionierung\@: Speicherplatz wird in sperate Datenbereiche geteilt
            \item NAND-Technologien\@: \\ -SLC (Single Level Cell): jede Zelle speichert ein Bit \\ -MLC (Multi Level Cell): jede Zelle zwei Bits \\ -TLC (Triple Level Cell): jede Zelle drei Bits \\ -QLC {Quadruple Level Cells}: jede Zelle vier Bits
        \end{itemize}
    \end{tcolorbox}

    \begin{figure}[H]
        \centering
        \includegraphics[width=0.8\textwidth]{./images/2.4.7_datentransferrate.png}
        \caption{Datentransferrate von Schnittstellen}\label{fig:Datentransferrate}
    \end{figure}

    \begin{subindent}
        NAS (Network Attached Storage) kann als Fileserverlösung benutzt werden um vor Ort Daten an mehreren Geräten zur Verfügung zu stellen.
        Alternativ können auch Cloud-Lösungen in Betracht gezogen werden. WLAN-Festplatten können verwendet werden wenn kein WLAN zur Verfügung steht.
    \end{subindent}

    \begin{tcolorbox}[width=15cm, center, title=NAS SAN und DAS, coltitle=white, colframe=orange, colback=white!60!orange]
        \begin{itemize}[itemsep=0.1em, parsep=0.3em]
            \item NAS\@: Network Attached Storage \\ eigenständiger Fileserver für ein Netzwerk
            \item SAN\@: Storage Area Network \\ Speichernetzwerk, dass mehrere Speicher verschiedener Orte zusammenfasst
            \item DAS\@: Direct Attached Storage \\ Speicher direkt und exklusiv mit einem Rechner verbunden
        \end{itemize}
    \end{tcolorbox}

\subsection{Tastaturen unterscheiden und präsentieren}
    \begin{subindent}
        Eine kurze Auflistung von verschiedenen Leistungskriterien, die betrachtet werden sollten bei Anschaffung einer Tastatur:
    \end{subindent}

    \begin{itemize}[leftmargin=2.5cm, topsep=0.2em, itemsep=0.1em, parsep=0.3em]
        \item Tastatur-Layout (z.B. QWERTZ in Deutschland)
        \item Tastenanzahl (78 ohne, 104 mit Num-Pad)
        \item Funk- oder Kabelgebunden, Virtuell
        \item Ergonomität (Handballenauflagen, Tastaturständer, Form)
        \item Haltbarkeit (Tastaturanschläge)
        \item Reinigungs-/Hygieneeigenschaften
    \end{itemize}

\subsection{Monitore vergleichen und präsentieren}
    \begin{subindent}
        Monitore gibt es in vielen verschiedenen Größen, Qualitäten, sowie Spezifikationen.
        Einen detailreiche Beschreibung aller Faktoren ist daher in diesem Umfang nicht möglich, und es wird nur sehr kurz zusammengefasst. \\
        \textbf{Bildschirmarten\@:}
        \begin{itemize}[leftmargin=2.5cm, topsep=0.2em, itemsep=0.1em, parsep=0.3em]
            \item PC-Monitor
            \item PC-Monitor mit TV-Funktion
            \item Touchscreen-Monitor
            \item Gaming-Monitor
            \item Multimedia-Monitor
            \item Curved-Monitor
        \end{itemize}
        \textbf{LCD-Display-Typen\@:}
        \begin{itemize}[leftmargin=2.5cm, topsep=0.2em, itemsep=0.1em, parsep=0.3em]
            \item TN\@: preisgünstig, schnelle Reaktionszeit, energiesparend
            \item VA\@: gute Bildqualität, etwas geringere Reaktionszeit
            \item IPS\@: sehr gute Bildqualität, teurer
            \item MVA, PVA
        \end{itemize}
        \textbf{ACM\@:} \\
        Adaptive-Contrast-Management, passt Kontrastverhältnis in sehr hellen/dunklen Szenen automatisch an \\
        Für Monitore gibt es Halterungen, mit denen man sie erhöht oder auch an der Wand befestigen kann. \\
        \textbf{Monitor-Technologien\@:}
        \begin{itemize}[leftmargin=2.5cm, topsep=0.2em, itemsep=0.1em, parsep=0.3em]
            \item LED\@: Light Emitting Diodes/Leuchtdioden \\ wandeln Strom in Licht um, Helligkeit proportional zur Stromstärke. Farben: RGB (Rot-Grün-Blau), durch Überlagerung auch andere
            \item LCD\@: Liquid Crystal Displays \\ zwei Lichtquellenvarianten: \\ 1. Edge-LED\@: sitzen im Rahmen des Monitors, Folie reflektiert nach vorne \\ 2. Direct-LED\@: sitzen direkt hinter Panel, leuchten nach vorne, schwarz dadurch dunkler
            %LED has more text
            \item OLED\@: Organic Light Emitting Diodes/organische Leuchtdioden \\ keine Hintergrundbeleuchtung notwendig \\ Strom wird durch Elektroden geschickt, wodurch die darin enthaltenen Halbleiterschichten leuchten 
            \item LED-Panel\@: LED-Leuchtfläche \\ preiswert, schnelle Reaktionszeit bis 1ms, gute Bildwiederholungsfrequenz
            \item TN-Panel\@: Twisted Nematic Panel, wie LED-Panel
            \item VA-Panel\@: Vertical Alignmetal \\ gute Kontrastwerte, natürliche Bilddarstellung
            \item IPS-Panel\@: In-Plane Switching \\ sehr gute Bildqualität/Farbgenauigkeit %weitere evtl adden
        \end{itemize}
        \textbf{Begriffe\@:} 
        \begin{itemize}[leftmargin=2.5cm, topsep=0.2em, itemsep=0.1em, parsep=0.3em]
            \item Bildwiederholungsfrequenz\@: Frames per Second (FPS), Anzahl der Bilder pro Sekunde
            \item Helligkeit\@: Candela (cd) = Lichtstärke, Einheit\@: $\frac{cd}{m^2}$ \\ für uns ist ein Wert von 250 bis 300$\frac{cd}{m^2}$ optimal
        \end{itemize}
    \end{subindent}

    %check why width=0.8 puts the img at the end of doc
    \begin{figure}[H]
        \centering
        \includegraphics[width=0.9\textwidth]{./images/2.4.9_screenspecs.png}
        \caption{Bildschirmspezifikationen}\label{fig:Bildschirmspezifikationen}
    \end{figure}

\subsection{Leistungsmerkmale für Drucker und Zusatzanforderungen erläutern}
    TODO
\subsection{Scanner beschreiben und für Arbeitsplatz auswählen}
    TODO
\subsection{IT-Zubehör für die Barrierefreiheit und im Aftersales unterscheiden}
    TODO
\subsection{Unternehmenssoftware anbieten und vergleichen}
    TODO
\subsection{Marktgängige IT-Systeme und Lösungen anbieten}
    TODO
\subsection*{Reflexion Kapitel 2.4}
\addcontentsline{toc}{subsection}{Reflexion Kapitel 2.4}
    %TODO REFLEXION
    \begin{refindent}
        TODO
    \end{refindent}